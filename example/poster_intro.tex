\section{\underline{Our main question:} \\ How to deal with strong inhibition?}

\inputcirc We have investigated \cheguei{how} to assess synchronization
of {\bf \color{blue} a neural population} whith {\bf \color{red} strong
  inhibition}

\usualsep\usualsep
\begin{center}
	\input tikz_pics/mainquestion.tex
\end{center}

\usualsep\usualsep
\hspace{1cm}
\begin{minipage}{0.26\textwidth}
  
  \inputcirc With \cheguei{Kuramoto oscillators}: \\ analytical
  {\color{blue} \bf synchronization threshold}.
  
\end{minipage}
\begin{minipage}{0.2\textwidth}
  \begin{flushright}
    \inputcirc With {\color{Green} \bf realistic models}: a novel \\ approach 
    through {\color{blue} \bf PRC learning}.
  \end{flushright}
\end{minipage}

\usualsep\usualsep


\section{Phase Resetting and Strong Inhibition}

\usualsep
\begin{center}
  \inputups {\color{blue} Biological patterns} generators are often
  described by a {\color{magenta} dynamical system} \\ trapped into a
  {\color{red} limit cycle} with a {\bf \color{Orange} huge} basin of attraction.
\end{center}

\usualsep\usualsep
\begin{minipage}{0.2\textwidth}
  \begin{center}
    
    \tikz{
      \node [rectangle, thick, draw=Green, font=\itshape]
            {\citacao{\footnotesize Izhikevich \& Ermentrout
                '2008, Scholarpedia (3:1487). } }; 
    }
    
    \usualsep
    \tikz{
      \node [rectangle, thick, draw=Green, font=\itshape]
            {\citacao{\footnotesize Canavier '2006, Scholarpedia
                (1:1332) } };
    }
    
    \usualsep
    \tikz{
      \node [rectangle, thick, draw=Green, font=\itshape]
            {\citacao{\footnotesize Krogh-Madsen et al '2012 } };
    }
    
    \usualsep\usualsep\usualsep
    {\color{blue}
    \inputarrow Phase Resetting curve (PRC):
    
    \vspace{0.3cm} 
    $ {\color{Green} \Delta \phi} = \lim\limits_{j\to\infty} \dfrac{j
      {\color{Black} T_0} - {\color{red} T_j}}{ {\color{Black} T_0} }
    \, \mbox{mod} \, 1 $
    }
    
    \usualsep\usualsep 
    \inputcirc It measures the {\bf \color{Green}
      phase difference ($\Delta \phi$)} after a pertubation is
    prompted!
  \end{center}
	
\end{minipage}
\begin{minipage}{0.25\textwidth}
  \begin{flushright}
    \inputups Firing pattern of LN neurons.
  \end{flushright}
  
  \includegraphics[scale = 2.5]{images/PhaseResetting_LN_editted.pdf}
\end{minipage}



\usualsep\usualsep
\begin{minipage}{0.22\textwidth}
    \includegraphics[scale = 0.8]{images/PRC_type12.png}
\end{minipage}
\begin{minipage}{0.26\textwidth}
  
  \inputcerto Positive \ \ $\Delta \phi$ means advance 
  
  \inputcerto Negative $\Delta \phi$ means delay
  
  \usualsep
  \inputcerto PRC {\color{Green} $\rightarrow$} Lyapunov exponent,
  prediction of synchronization and phase locking.
  
  \usualsep\usualsep
  \inputups Nevertheless, pulse must be {\bf \color{red} brief and weak}!
  
\end{minipage}


\usualsep\usualsep\usualsep
\begin{center}
  \inputarrow And apparently {\bf \color{blue} strong inhibition} is present in nature
\end{center}

\usualsep\usualsep
\begin{minipage}{0.28\textwidth}
  \includegraphics[scale = 1.2]{images/natborggraham.png}
\end{minipage}
\begin{minipage}{0.22\textwidth}
  \begin{center}
    
    \tikz{
      \node [rectangle, thick, draw=Green, font=\itshape]
            {\citacao{\footnotesize Borg-Graham et al ' 1998, Let. to Nature}};
    }
    
    \usualsep
    \tikz{
      \node [rectangle, thick, draw=Green, font=\itshape]
            {\citacao{\footnotesize Root et al '2008, Cell (59:311)}};
    }
    
    \usualsep    
    \tikz{
      \node [rectangle, thick, draw=Green, font=\itshape]
            {\citacao{\footnotesize Folias et al ' 2013, Euro. J. Neurosci. (38:2864)}};
    }
    
    \usualsep\usualsep 
    
    \inputarrow Also, an collaborator recently observed strong
    currents in a hyphotalamic circuit (unpublished).
    
  \end{center}
  
\end{minipage}


\usualsep\usualsep\usualsep\usualsep\usualsep\usualsep\usualsep\usualsep\usualsep
\phantom{Breu!}
