\section{Predicting Synchronization with Kuramoto Oscillators}

\begin{minipage}{0.23\textwidth}
  {\includegraphics[scale=1.3]{images/groups.pdf}}
\end{minipage}
\begin{minipage}{0.25\textwidth}
  
  \begin{center}
    \tikz { \node [rectangle, thick, draw=Green, font=\itshape]
  {\citacao{\footnotesize Amig\'o, Mosqueiro and Huerta '2014, {\bf
        Submitted} } }; }
  \end{center}
  
  \usualsep\usualsep
  \inputups {\color{blue} Inspired by the {\bf striatum} architecture}
  
  \usualsep  
  \inputups {$N$ oscillators per group}
  
  \usualsep
  \begin{flushright}
    \inputcerto {\color{Red} 
      $j$-th kuramoto oscillators i group $n$: 
      
      \begin{center}
        $ \dot{\phi}_j^n (t) = \omega - \epsilon F^{n-1}(t) $
      \end{center}
    }
    
    \color{Orange}
    $ F^{n}(t) = \frac{1}{N} \sum\limits_{j=1}^N \chi_{[2\pi-\Delta,2\pi)}\left[ \phi^n_j(t) \right] $
  \end{flushright}
  
\end{minipage}

\usualsep\usualsep
\begin{minipage}{0.253\textwidth}
  
  \begin{center}
    
    \inputcerto We have found a synchronization threshold:
    
    \usualsep
    \tikz{
      \tikzstyle{quadro} = [rectangle, rounded corners = 10mm, minimum
        size=5cm, line width = 2mm, draw=blue, top color = white,
        bottom color=Blue!20]
      \node [quadro] (mean) at (-8,-6.5) { \ \ \ \  $ \epsilon_{th} \, = \,
        \dfrac{2\pi \omega}{\Delta} $  \ \ \ \ };
    }
  \end{center}
  
  \begin{center}
    \includegraphics[scale = 2.0]{images/PhaseTransition.pdf}
  \end{center}
\end{minipage}
\begin{minipage}{0.22\textwidth}
  \begin{flushright}
    \includegraphics[scale = 2.5]{images/Phases_vs_Time_k10.pdf}
  \end{flushright}
  
  \inputups Example of phases over time per group precisely at the boundary
\end{minipage}


\section{Extending results to realistic models: \\ a Learning Approach}

\inputarrow Using the same circuit, we want to undestand the
synchronization of LN neurons

\usualsep
\begin{minipage}{0.2\textwidth}
  
  \inputcirc {\color{blue} 16 differential equations}
  
  \inputcirc 2 comportaments ({\bf \color{Green} soma + axon})
  
  \usualsep
  \inputcirc Integration with CVODE
  
\end{minipage}
\begin{minipage}{0.25\textwidth}
  
  \begin{center}
    
    \inputcirc With {\color{red} $20$ neurosn per group}, simulatiosn
    takes at least {\color{Orange} $3$ hours}
    
    \usualsep
    \inputcirc In contrast to Kuramoto model:
    
    {\color{blue} some minuts...}
    
  \end{center}
\end{minipage}

\usualsep\usualsep
\begin{minipage}{0.2\textwidth}
  
  \inputcirc {\color{blue} The idea: let the Kuramoto oscillators
    learn the PRC of LN neurons}
   
\end{minipage}
\begin{minipage}{0.25\textwidth}
  
  \begin{center}
    
    {\color{blue} $\dot{\phi}_j^n(t) = \omega - {\color{magenta}
        \epsilon \left[ \phi(t) \right]} F^{n-1}(t)$}
    
    \usualsep
    \inputarrow {\color{magenta} $\epsilon \left[ \phi(t) \right]$} is the desired PRC
    
  \end{center}
\end{minipage}

\usualsep\usualsep\usualsep\usualsep\usualsep\usualsep\usualsep\usualsep\usualsep
here...
